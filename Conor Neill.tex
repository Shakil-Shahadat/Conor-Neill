\documentclass[a4paper,12pt]{article}

% Set Border
\usepackage[a4paper, total={6.5in, 9.5in}]{geometry}

% Link Package
\usepackage{hyperref}
\hypersetup{
	colorlinks=true,
	linkcolor=blue,
	filecolor=magenta,
	urlcolor=black
}

% Highlight
\usepackage{xcolor, soul}
% Set highlight color
\sethlcolor{green}



\begin{document}

\title{ \textit{Transcripts \& Notes of Videos of Conor Neill}  }
	\author{ by Shakil Shahadat }
	\date{ \today }
\maketitle

% No page number in the first page
\thispagestyle{empty}

% End Current Page
\newpage




\section{ \href{https://www.youtube.com/watch?v=5dC5g7wgxOA}{On Taking Intelligent Action... and Improving your Intuition} [ 13 December, 2024 ] }


Last week's video was, \hl{life rewards action more than intelligence.} And that video has been one of my most successful videos of the last few years in terms of the engagement and the number of viewers watching it on YouTube, on LinkedIn. And this is just a moment to comment on some of the questions, the comments that have come in from that video. [ 0:25 ] \\

I think many of the comments were extremely positive saying, absolutely Conor, I needed to hear this. This is so true. For many years, I stayed in my intelligence and waited for the world to discover me and I discovered that \hl{only action begins the change.} [ 0:42 ] \\
 
But there were quite a number of comments that said, well, dumb action, isn't dumb action worse than any action at all? And I think, the answer is absolutely yes. \hl{Dumb action is probably worse than no action at all.} [ 0:55 ] \\

But I would say, the watchers of these videos, the community that comes to my channel on YouTube or watches these videos on LinkedIn, the likelihood is, you're intelligent enough to not take dumb action. Or at the very least if you did take dumb action to see it very quickly and readjust and change the direction of your action. [ 1:21 ] \\

So, for each of you, the worry is definitely not taking dumb action. For each of you, \hl{the biggest barrier is taking the good ideas that you have and turning them into the first action that makes an impact.} And I think there's two reflections that I wanted to share with you just on this idea of taking better action. [ 1:35 ] \\

So, clearly, \hl{taking any action is better than never taking any action.} But once you start to take action and be systematic about daily making progress on taking actions, now, \hl{you do need to tune your intuition to take better decisions and where you focus your time for the greatest impact.} [ 1:53 ] \\

And for me, there's a book by Peter Drucker called `Managing Oneself' that makes a recommendation that I see so few people do. And his recommendation is, \hl{you get a journal and when you take a decision, at the time you take the decision, you write down why you took the decision and what you predict is going to happen.} You write down your prediction and it might be a prediction a month out, a prediction a year out, a prediction five years out. But in each time you take an important action or you take an important decision, he recommends you stop and you write down why you took it and what you think is going to happen. And then 3 months later, a year later, when the time comes to look at your prediction, you force yourself to look at what you expected was going to happen and what actually happened. [ 2:46 ] \\

And Peter Drucker says, \hl{the only way you're going to improve your intuition is through this process of being forcing yourself to write down your prediction and then forcing yourself to look at whether your prediction and what really happens are the same.} [ 3:03 ] \\

And that, \hl{this feedback is what's going to allow you to tune your intuition to improve the quality of your action, the quality of your decisions.} And in my own case, selecting people is one of the most important roles that I have as a leader in my organization. [ 3:21 ] \\

And knowing this, that, from an hour, I need to decide on people that will then go on and will they do the work and I started writing down my predictions, writing down what I thought the risks are, writing down what I thought was going to happen and a year later, really looking at the data, how did it turn out and I learned a lot about how poorly I was doing and interviewing process, how poorly I was looking into one of the real factors that determine whether someone is going to be successful in the role or not. [ 3:58 ] \\

And it was only through this very painful and harsh feedback. Because I liked the thought that I was a good judge of people. But, when I started writing down in my journal, my prediction of who would have a massively positive impact, who would have struggled to have it happen and who I really didn't see. I starting to see I wasn't picking up in my process of interviewing that the clues that would really allow me to take good people decisions. \\

[ 4:26 ] And by having that data, by seeing, it was first very hard for me to confront that I was not being a good judge of people. But it forced me to go back and look at what I was paying attention to, look at what questions I was asking, look at how I was conducting an interview and make some changes so that I'm getting better at being able to take good people decision. \\

[ 4:53 ] So, this idea from Peter Drucker and I'll add a link up here to some material from Peter Drucker on how to know yourself, how to know what you're good at, how to know what you're not good at. \\

[ 5:04 ] But one of this thing is what he calls feedback analysis, which is, when you take action, when you take a decision, note down on paper what you expect to happen and then months later, make sure you look at reality and what you've predicted. \\

[ 5:19 ] And this is the only way, your intuition has a chance of improving and being better at taking the decisions, taking the action that can really move your life forward. \\

[ 5:28 ] So, anyway, a reflection on not just taking any action but improving day after day, week after week, the quality of your action, the focus of your action, the way in which you conduct the action. But, clearly, from nothing to any action that's a good step. But then if you have a feedback loop that allows you to improve each day, the quality of your action, even better. \\

[ 5:50 ] So, anyway, it's Conor here from ESA business school. Been teaching a global program today. Hope you have a great Christmas period. Christmas coming up for those of you who celebrate it. I'll be heading over to Ireland in about a week's time to spend it with family. Have a great Christmas time. Thank you for your likes, your shares, your comments. The channel has reached 400,000 subscribers today. So, again, I want to thank each of you for your sharing, your contribution, being part of it. It inspires me to keep pulling out the camera, making these videos and sharing these ideas with the world. Have a good one. \\



\subsection*{ Summary }

\begin{enumerate}

	\item Life rewards action more than intelligence.

	\item Only action begins the change.

	\item Dumb action is worse than no action at all.

	\item The biggest barrier is taking the good ideas that you have and turning them into the first action that makes an impact.

	\item Taking any action is better than never taking any action.

	\item You need to tune your intuition to take better decisions and where you focus your time for the greatest impact.

	\item Get a journal and each time you take an important action or an important decision, you write down why you took the decision and what you predict is going to happen. Later when you get a result of that action / decision, compare that with what you predicted.

	\item The only way you're going to improve your intuition is through this process of forcing yourself to write down your prediction and then forcing yourself to look at whether your prediction and what really happens are the same.

	\item This feedback is what's going to allow you to tune your intuition to improve the quality of your actions / decisions.

\end{enumerate}



\subsection*{ Links }

\begin{enumerate}

	\item Book recommendation: \href{https://www.goodreads.com/book/show/2477223.Managing_Oneself}{Managing Oneself by Peter F. Drucker}.

\end{enumerate}


\subsection*{ Comments }

\begin{enumerate}

	\item Writing down the expectations and then comparing them later with the reality is one type of feedback loop or reinforcement learning.

	\item What was the real factor mentioned after 3:21?

\end{enumerate}



% End Current Page
\newpage





\end{document}
